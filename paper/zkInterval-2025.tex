% IACR Communications in Cryptology template file
% This file shows how to use the iacrcc class to write a paper.

%% Document mode
% [version=xxx] where xxx is preprint, submission, or final. The default is preprint.
%               version=submission puts line numbers into the document and makes it
%               anonymous (which can be overridden).
%               version=final is for submitting the final version. This requires a license
%               to be specified.
% [notanonymous]  Keep author names in submission mode
%% Package options
% [floatrow]      Load floatrow package with correct captions
% [biblatex]      Use the biblatex package instead of bibtex. Note that options
%                 may not be passed to biblatex at this time.

% Uncomment the next line if you have TeX Live 2021 or later
% and want to produce a PDF which complies with the PDF/A-2U standard
% \DocumentMetadata{pdfstandard=a-2u}
\documentclass{iacrcc}

% When the *final* document mode is used
% the authors need to provide a supported license.
% In all other modes this information is ignored.
% The currently provided ones are { CC-by }
\license{CC-by}

% Include LaTeX packages required by your paper
% \usepackage{}

% Provide the title of the paper
% This should look like:
\title[running  = {zkInterval},
      ]{zkInterval: Trustless, Confidential, Constant-Size Range Proofs}
% Where the options in square brackets “[ ]” are optional and control the following:
% running:   the running title displayed in the headers
% subtitle:  provide a subtitle
% plaintext: a text version of the title (mandatory if macros are used in the title)

% Define authors and affiliations
% Authors are listed individually using the \addauthor tag followed by a list of affiliations.
% The idea is that every author makes a separate call to this command.
% This should look like:
% \addauthor[inst      = {1,2},
%            orcid     = 0000-0000-0000-0000,
%            footnote  = {Thanks to my supervisor for the support.},
%            onclick   = {https://www.mypersonalwebpage.com}
%           ]{Alice Accomplished}
% Where the options in square brackets “[ ]” are optional 
% and control the following:
% inst:     a numerical list pointing to the index of the institution 
%           in the affiliation array.
% orcid:    create a small clickable orcid logo next to the authors name 
%           linking to the authors ORCID iD see: orcid.org.
% footnote: create an author-specific footnote.
% surname:  indicate the surname of the author for indexing purposes.
% onclick:  define what to do when clicking on the external link logo
%           next to the author name: e.g., can point to the academic webpage.
% email:    define the e-mail address of this author.
\addauthor[inst    = {1},
           onclick = {https://www.logicalmechanism.io/},
           email   = {quinn@logicalmechanism.io},
           surname = {Parkinson}
          ]{Quinn Parkinson}

% The following command controls the running header for authors
% This is optional for <= 4 authors and mandatory for > 4 authors
% \authorrunning{Joppe W. Bos and Kevin S. McCurley}

% Affiliations are listed individually using the \addaffiliation command 
% *after* the (list of) authors using \addauthor
% This should look like (full example):
% \addaffiliation[ror        = 05f950310,
%                 department = {Computer Security and Industrial Cryptography},              
%                 street     = {Kasteelpark Arenberg 10, box 2452},
%                 city       = {Leuven},
%                 state      = {Vlaams-Brabant},
%                 postcode   = {3001},
%                 country    = {Belgium}
%                ]{KU Leuven}
% Where the options in square brackets “[ ]” are optional and control 
% the following (optional information is mainly used for meta-data collection):
% ror:        provide the Research Organization Registry (ROR) identifier 
%             for this affiliation (see: ror.org). This is used for meta-data 
%             collection only.
% department: department or suborganization name
% street:     street address
% city:       city name
% state:      state or province name
% postcode:   zip or postal code
% country:    country name. Required for [version=final]
\addaffiliation[ country = {USA}
               ]{Logical Mechanism LLC}

% Authors should use the \addfunding macro to make sure that funding agencies
% can find papers published under their sponsorship.
% You can use the online tool at
% https://publish.iacr.org/funding
% to help you find fundref and ror identifiers.
% Note that \addfunding *does not* automatically create footnotes or
% an acknowledgements section to identify funding - it only collects the
% metadata for indexing.
% An example is:

% \addfunding[country  = {europe},
%             grantid  = {1234},
%             fundref = {100010661}
%            ]{Horizon 2020 Framework Programme}

% A footnote can be placed on the front page without a symbol / numbering using:
%\genericfootnote{This is the full version of our paper published at XX}

\begin{document}

\maketitle

% Provide the abstract of your paper
\begin{abstract}
  In this paper, we introduce zkInterval, a novel constant-size range proof protocol that proves a secret value lies within a public interval without revealing the value or requiring a trusted setup. Leveraging Pedersen commitments, Weil pairings, and Schnorr $\Sigma$-protocols, zkInterval offers a practical, trustless, and confidential solution with proofs whose size remains invariant regardless of the interval's range. These properties significantly reduce complexity, storage, and communication overhead, making zkInterval particularly well-suited for privacy-focused applications and scenarios with variable interval lengths. Our protocol enhances the efficiency and scalability of cryptographic range proofs, offering a robust solution to a critical challenge in modern cryptography.
\end{abstract}

% A separate text-only abstract must be supplied in your final version.
% This will be used for web pages and indexing and should not contain macros.
\begin{textabstract}
\end{textabstract}

% The content of the paper starts here
\section{Introduction}

Proving that a value lies within a specific interval while preserving its confidentiality practically and efficiently poses a significant challenge in modern cryptographic range proofs. Many existing methods generate proofs whose size scales with the interval's range, which can severely limit their efficiency, especially in applications that demand a constant proof size across variable interval lengths or must avoid trusted setups, which can introduce vulnerabilities such as collusion-based attacks. This paper introduces zkInterval, a protocol that overcomes these limitations by producing a practical, trustless, and constant-size range proof independent of the interval's range. By integrating Weil pairings \cite[pp.62--63]{menezes1993}, Pedersen commitments \cite{pedersen1991}, and Schnorr $\Sigma$-protocols \cite[pp.310--311]{thaler2022}, zkInterval securely demonstrates that a secret value lies within a publicly known interval while keeping the value itself hidden. Our efficient and concise approach makes zkInterval particularly well-suited for privacy-focused applications with variable interval lengths. Since privacy-preserving range proofs play an essential role in private information verification, secure multi-party computation, and confidential transactions, zkInterval addresses a significant need in modern cryptographic systems.

Although some schemes offer constant-size proofs, they often rely on trusted setups or result in verbose proofs \cite{cryptoeprint:2024/430}. More recent advancements like Bulletproofs \cite{bunz2018bulletproofs} have reduced proof sizes to sublinear scales, all while being trustless; however, their proof size still depends on the interval's range, which can be a limiting factor in applications with variable interval lengths. zkInterval addresses these challenges by delivering constant-size, scalable range proofs that maintain confidentiality without compromising efficiency. This work fills a critical gap in current cryptographic methods and paves the way for further advancements in range-proof technologies.

The remainder of this paper is as follows. Section 2 presents the derivation of the zkInterval protocol, detailing its mathematical foundation and the integration of Weil pairings, Pedersen commitments, and Schnorr $\Sigma$-protocols. Section 3 rigorously examines the protocol's security properties, establishing its zero-knowledge guarantees and demonstrating that it forms a complete proof system. Finally, Section 4 discusses the practical implementation of zkInterval on the Cardano blockchain, highlighting its performance benefits and suitability for privacy-focused and scalable cryptographic applications.

\section{Bibliography}
Citing papers is done in the usual way using BibTeX or \texttt{biblatex}
commands. For example: the RSA paper~\cite{RSA78}.

It is highly encouraged to use CryptoBib from \url{https://cryptobib.di.ens.fr}

% This sample uses bibtex rather than biblatex.
\bibliography{references}

% NOTES
% - Download abbrev3.bib and crypto.bib from https://cryptobib.di.ens.fr/
% - Use biblio.bib for additional references not in the cryptobib database.
%   If possible, take them from DBLP.

\end{document}
